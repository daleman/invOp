\section{Pautas de Implementación}
\par{Al igual que en trabajos prácticos anteriores, desarrollaremos
nuestras soluciones en \texttt{C++} y recurriremos a la
\texttt{STL} del mismo de ser necesario. Al igual que en el trabajo práctico
anterior, usaremos una clase grafo propia que representa un grafo mediante
listas de adyacencia y matriz de adyacencia simultáneamente. En el caso de
precisar implementar alguna otra estructura de datos específica vamos a
documentar las características de sus operaciones y adjuntar el código de estas
en el apéndice del código de este informe. Los programas desarrollados se
encuentran en el directorio \texttt{/codigo} dentro del archivo
\texttt{entregable.tar.gz}, los cuales son:}

\begin{itemize}
\item \texttt{exacto/exacto.h} - Contiene la implementación del algoritmo exacto (II).
\item \texttt{goloso/goloso.h} - Contiene la implementación de la heurística golosa (III).
\item \texttt{local/local.h} - Contiene la implementación de la heurística de búsqueda local (IV).
\item \texttt{tabu/tabu.h} - Contiene la implementación de la metaehurística de búsqueda tabú (V).
\item \texttt{timer.h} - Contiene funciones para medir performance.
\item \texttt{grafo.h} - Contiene la clase grafo utilizada en todos los algoritmos.
\end{itemize}

\par{En este directorio también tenemos un \texttt{Makefile} que permite
compilar separadamente cada implementación. De esta manera, si quisiera correr
el algoritmo exacto con los casos de prueba del archivo \texttt{tests.in},
debría compilarlo mediante \texttt{make EXACTO} y luego correrlo escribiendo\\
\texttt{./exacto <  tests.in}.}\\

\par{Para testear los tiempos de ejecución de las
implementaciones contamos con los programas
$test\_exacto$, $test\_goloso$, $test\_local$ y $test\_tabu$ cada uno
en su respectiva carpeta. Cada uno se ejecuta con los parámetros
$N$, $t$ y $s$. Para cada $n$ = $k*s$ ($k$ entero) menor o igual a
$N$, se generan $t$ instancias del problema de tamaño $n$ y se
ejecutan las soluciones implementadas con esas instancias. Luego
se devuelven los resultados obtenidos y los tiempos de ejecución
promedio en función de $n$.}\\

\par{Para testear la optimalidad de las implementaciones contamos con los
programas $opt\_goloso$, $opt\_local$ y $opt\_tabu$ cada uno en su respectiva
carpeta. Estos se ejecutan con los mismos parámetros que sus versiones $test$
asociadas, pero en lugar de devolver los timepos de ejecución, devuelven
lo que llamamos el índice de optimalidad, que nos dice que tan cerca estuvo
el resultado devuelto por la heurística del resultado óptimo. El índice de
optimalidad es un número menor o igual a cero que se calcula como $f$-$O$
con $f$ el valor de la solución devuelta por la heurística y $O$ el valor
de la solución óptima\footnote{A último momento se modificó el índice
de optimalidad para que sea un porcentaje respecto al óptimo en lugar
de ser sólo la diferencia con él.}.}\\

\par{Una instancia del problema de $CMF$ es un grafo cualquiera. La clase
grafo cuenta con 2 constructores: uno sin parámetros que genera un grafo
vacío y otro con un único parámetro $n$ que genera un grafo de $n$ nodos
sin aristas. La función $inicializar$ de la clase $grafo$ toma un grafo vacío
y lo reconstruye a partir de la entrada estándar. La función
$generar\_aristas\_aleatorias$ toma un grafo de $n$ nodos y, por cada posible
arista del grafo (es decir, por cada par de nodos) decide aleatoriamente
si agregarla al grafo o no. La función $generar\_m\_aristas\_aleatorias$
funciona igual pero genera un grafo con exactamente $m$ aristas ($m$ parámetro).}\\

\par{Una solución al problema de $CMF$ es un subconjunto de nodos del
grafo entrada. Para representar una solución utilizaremos una dupla
de un entero y un vector de $n$ booleanos (con $n$ la cantidad de nodos
del grafo). El i-ésimo booleano del vector indica si el i-ésimo nodo del
grafo pertenece a la clique representada por la solución. El entero de la
tupla corresponde al tamaño de la frontera de la solución representada. Notar
que no toda solución representada de esta forma es una solución válida, ya
que no necesariamente representa una clique. A las soluciones que no sean
cliques se les asigna como segundo elemento de la dupla el valor de la
frontera de su subconjunto de nodos\footnote{La frontera no sólo está
definido para cliques sino que lo está para cualquier subconjunto de nodos
de un grafo.}
multiplicado por -$1$, así las soluciones inválidas tendrán
valor de frontera negativo.}
