\section{Introducción}
\par{El presente informe apunta a documentar el proceso de desarrollo del Trabajo
Práctico de la materia Investigaci\'on Operativa, cursada
correspondiente al segundo cuatrimestre del año 2013.}\\

\par{Este trabajo práctico consiste en el análisis de diferentes algoritmos
para problemas de programcion lineal entera, para encontrar el conjunto independiente maximo de
un grafo y evaluar los resultados de los distintos algoritmos sobre instancias de benchmark utilizadas en la literatura e instancias generadas aleatoriamente.

 Dado un grafo $G=(V,E)$, un conjunto independiente de $G$ es un subconjunto $K$
de los nodos de $V$ tal que para todo par de nodos de $K$, no existe una arista
en $E$ que los une. Es decir,}
\[
sea\ G = (V,E),\ K \subseteq V\ es\ conjunto\ independiente\ de\ G \Leftrightarrow (v,w)
\notin E\ \forall\ v,w \in K
\]

Para el desarrollo de los algoritmos se utilizar\'an las librer\'ias que provee el paquete CPLEX,
en particulae la Callable Library.
Los algoritmos Cut and Branch y Branch and Cut requieren el desarrollo de algoritmos de
planos de corte, que incluyen las siguientes desigualdades v\'alidas:

\begin{itemize}
\item Cortes Clique
\item Cortes Ciclo impar
\end{itemize}
%\par{Exhibimos un ejemplo del problema en la siguiente figura:}

%\textbf{GRAFICO$\_$COPADO.PNG}

%\par{Podemos ver como los vértices coloreados con negro corresponden a la
%clique que tiene la frontera mas grande del grafo, mientras que los coloreados
%con gris corresponden a su frontera.}\\

\par{Para resolver este problema se pide implementar:}
\begin{itemize}
\item Un algoritmo branch and bound.
\item Un algoritmo branch and cut.
\item Un algoritmo cut and branch.
\end{itemize} 

\par{Luego de haber terminado con la programación de los algoritmos ,
se realizar\'a' una experimentación con el objetivo de verificar y comparar
tanto el gap como los resultados de los programas extrayendo
así conclusiones sobre la performance y la optimalidad de los diferentes
métodos propuestos en este trabajo práctico.}\\

\par{Para comenzar se describir\'a el modelo de un conjunto independiente que se eligi\'o para 
las implementaciones:

$$ \begin{array}{lll}
    \textbf{max} & \displaystyle \sum_{j=1}^{n} x_j \\
    \textbf{s.a} & \displaystyle x_i + x_j \leq 1 & \forall i,j \in E(G) \\
                 & \displaystyle x_j\geq 0 & j\in [1..n]
\end{array}$$ 








}
