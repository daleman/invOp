Para los algoritmos de branch and cut y de cut and branch, se necesitar\'an planos de corte, que tengan desigualdades v\'alidas. En este trabajo se realizar\'an los cortes clique y los cortes de ciclo impar.

\section{Generador de Cliques}

\par{Para generar los los cliques, se implemento una heuristica que genera cliques Maximales,
(no necesariamente \'optimos). La heur\'istica consiste en:
Teniendo un conjunto de nodos tales que se pueden conectar con la clique actual (que se inicializa con todos los nodos adyacentes a un nodo), se van agregando nodos que pertenecen a este conjunto y se elimina del conjunto a los nodos no adyacentes al reci\'en agregado.
As\'i obtenenmos una clique, ya que todos los nodos son adyacentes de a pares, que es maximal (ya que dejo de agregar nodos cuando el conjunto de $seConectanConCliqueActual$ es vac\'io).

Los cortes clique son muy \'utiles a la hora de agregar desigualdades al problema debido a que restringen m\'as a las posibles soluciones del problema generando as\'i podas en el \'arbol de soluciones.

Debido a que en una clique todos los nodos son adyacentes de a pares, si alguno de los nodos de esa clique, 
pertenece a un conjunto independiente de G, luego ningun nodo de esa clique pertenece a ese conjunto independiente, es decir:

\begin{equation*}
    \forall K \sum_{i \in K} x_i \leq 1 \\
\end{equation*}
con K una clique de G.

\section{Generador de Ciclos Impares}

\par{Otro tipo de cortes que se generaron en este trabajo son los cortes de Ciclos Impares.
En el grafo se encontraron ciclos impares haciendo dfs ($depth first search$) y cada vez que encuentro un ciclo, se fija si es impar.
Como se recorre el grafo en profundidad, es probable que los ciclos que encuentre sean de muchos nodos.

El corte del ciclo impar se modela de la siguiente manera:

\begin{equation*}
    \forall C \sum_{i \in C} x_i \leq \frac{\left\vert{C}\right\vert -1}{2}  \\
\end{equation*}
con C un ciclo impar de G.


}

\newpage
\section{Experimentación}

Se generaron tres tipos de algoritmos:
\begin{itemize}
\item Un algoritmo Branch and Bound.
\item Un algoritmo Branch and Cut.
\item Un algoritmo Cut and Branch.
\end{itemize} 

Es preciso aclarar que para los tres algoritmos se deshabilitaron todos los tipos de corte y preprocesamiento
proporcionado por CPLEX.\\

\textbf{Algoritmo Branch and Bound:} Este algoritmo resuelve el problema aplicando sucesivamente
el algoritmo simplex en la relajaci\'on lineal del modelo. Luego elige una variable $x_j$ que no es entera y 
llama a la funcion recursiva para seguir resolviendo de dos maneras, con $\left \lceil{x_j}\right \rceil$ y 
con 
$\left \lfloor{x_j}\right \rfloor$ hasta que la soluci\'on de la relajaci\'on lineal sea entera.\\


\textbf{Algoritmo Branch and Cut:} Este algoritmo resuelve el problema aplicando sucesivamente
el algoritmo simplex en la relajaci\'on lineal del modelo. Luego evalua los cortes proporcionados
(los cortes clique y los de ciclo impar) y si hay alg\'una desigualdad violada se agrega a las restricciones.
Por como est\'a implementado se pude agregar solo algunas restricciones por cada iteracio\'on, en especial 
se agregan las que violan la desigualdad con mayor diferencia.
Posteriormente como en branch and bound, se elige una variable $x_j$ que no es entera y llama 
a la funci\'on recursiva para seguir resolviendo de dos maneras, con $\left \lceil{x_j}\right \rceil$ y con 
$\left \lfloor{x_j}\right \rfloor$ hasta que la soluci\'on de la relajaci\'on lineal sea entera.\\

\textbf{Algoritmo Cut and Branch:}
Se basa en aplicar los cortes unicamente en el nodo ra\'iz y luego ejecutar un Branch and Bound.\\

\newpage
\section{Resultados}

Los algoritmos tienen tres par\'amtros. El primero es la cantidad de cliques que se buscan, el segundo
la cantidad de ciclos imapres. El tercer par\'ametro es el l\'imite de cortes que se pueden agregar por cada 
iteraci\'on. En el caso de que se encuentren menos cliques o ciclos impares que los pasados por par\'ametro
se busca la mayor cantidad posible.

En la pr\'oxima tabla se muestra los resultados sobre la instancia frb30-15-1.mis \footnote{grafo de
http://www.nlsde.buaa.edu.cn/~kexu/benchmarks/graph-benchmarks.htm}.

\begin{center}
\begin{tabular}{|l|l|l|l|l|l|}
\hline
Algoritmo & Parametros & Gap nodo Raiz & Gap Nodo FInal & Solucion & #nodos\\ \hline
Branch and Bound & × & 971 & 800 & 25 & 865148\\ \hline
Branch and Cut & 100 100 10 & 643 & 548 & 27 & 71162\\ \hline
Branch and Cut & 100 100 200 & 681 & 606 & 25 & 58921\\ \hline
Branch and Cut & 450 12 460 & 741 & 505 & 25 & 49559\\ \hline
Branch and Cut & 100 100 5 & 938 & 667 & 23 & 54314\\ \hline
Cut and branch & 500 12 512 & 741 & 505 & 25 & 49559\\ \hline
Cut and branch & 200 12 212 & 751 & 394 & 25 & 1233045\\ \hline
\end{tabular}
\end{center}

En la proxima tabla se muestra los resultados sobre la instancia dsjc_250_5.in \footnote{grafo de dsjc\_250\_5 
de http://www.info.univ-angers.fr/pub/porumbel/graphs/} .

\begin{center}
\begin{tabular}{|l|l|l|l|}
\hline
Algoritmo & Parametros & Gap Nodo Final & Solucion\\ \hline
Branch and Bound & × & 638 & 12\\ \hline
Branch and Cut & 100 100 10 & 418 & 12\\ \hline
Branch and Cut & 100 100 200 & 0(fin en 3128s) & 12\\ \hline
Branch and Cut & 100 100 5 & 364 & 12\\ \hline
Cut and branch & 500 2000 2500 & 0(termina en 2176) & 12\\ \hline
Cut and branch & 200 2000 2200 & 0(fin 2216s) & 12\\ \hline
\end{tabular}
\end{center}


\section{An\'alisis de resultados y Conclusiones}


\par{
En primer lugar se puede observar la diferencia entre Branch and Bound y Cut and Branch.\\

Como era esperable el gap en el nodo raiz en Branch and Bound es mayor que el de los dem\'as algoritmos ya que
no cuenta con los cortes.

Si bien en en Cut and Branch solamente se aplican los cortes en el nodo raiz, 
en el test random se observa que es m\'as eficiente y por lo general el gap en el \'ultimo nodo (cuando no termina
el agoritmo) es mucho menor en comparaci\'on con branch and bound.\\

Comparando Branch and Bound con Cut and Branch, se ve (en el test random) que el Cut and Branch es m\'as 
eficiente, teniendo por lo general un gap mas chico en el nodo final. Igualmente la diferencia no es tan
grande como con Branch and Bound.

}



