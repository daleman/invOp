\section{Conclusiones}
En primera instancia, la realizaci\'on del algoritmo exacto nos sirvi\'o para entender que el conjunto de soluciones
posibles era exponencial sobre la cantidad de nodos del grafo.
Luego comenzamos a implementar las heur\'isticas.\\

La heur\'istica golosa, fue una primera aproximaci\'on a encontrar una soluci\'on,
no necesariamente \'optima con una cantidad polinomial de operaciones. A pesar de que la idea de la heur\'istica es muy sencilla
(agregar a la soluci\'on nodos que se conecten con la clique actual y que aporten la maxima frontera posible), pudimos obtener una cota
inferior :\\
La frontera siempre ser\'a mayor o igual al grado m\'aximo del grafo.\\


Con la heur\'istica local pudimos obtener soluciones mejores que las que proporcionaba la golosa (si la vecindad era mayor que 1) con
uan eficiencia en cuanto a cantidad de operaciones comparado con el algoritmo exacto.\\


Finalmente con la heuristica tabu, obtuvimos soluciones que generalmente difieren
en un \%10 de la soluci\'on \'optima ( \%5 si, el par\'ametro de vecindad es igual a 1 ). Esto es gracias a que recorre muchas soluciones
y no para s\'olamente si la soluci\'on es m\'axima de forma local.\\


Con el trabajo pr\'actico pudimos entender las ideas b\'asicas de las heur\'istcas que son esenciales para problemas en los que los algoritmos exactos
conocidos son exponenciales. 
