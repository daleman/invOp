\section{\'Apendice}

\subsection {Grafo y sus funciones}
\begin{lstlisting}

bool funcionSort (vector <int> i, vector <int> j) { return ( i.size() > j.size() ); }


struct arista{
    int nodo1;
    int nodo2;

    arista(int n, int m){
        this->nodo1 = n;
        this->nodo2 = m;
    }

};

class grafo{
    public:
        int cantNodos;
        vector< vector<int> > lista_global;
        vector< vector<int> > matriz_ady;
        vector<arista> aristas;

        //Constructor: Construye un grafo vacío
        grafo(){
            grafo(0);
        }

        //Constructor: Construye un grafo con n nodos sin aristas
        grafo(int n){
            constructor(n);
        }

        //Construye el grafo a partir de la entrada estándar
        // (llamar después del constructor vacío)
        void inicializar() {
            int n, m, a, b, c;
            cin >> n;
            constructor(n);
            if(n != 0) {
                cin >> m;
                for(int i = 0; i < m; ++i){
                    cin >> a;
                    cin >> b;
                    asociar(a,b);
                }
            }
        }

        //Construye un grafo con aristas aleatorias
        // (llamar después del constructor básico)
        void generar_aristas_aleatorias() {
            //cambio esta funcion para que genere un grafo completamente random
            srand(time(NULL));
            for(int i = 0; i < this->cantNodos; ++i)
                for(int j = i; j < this->cantNodos; ++j)
                    if(rand()%2 == 0) asociar(i, j);
        }

        void asociar(int nodo1, int nodo2){
            
            nodo1 = nodo1-1;
            nodo2 = nodo2-1;
            
            if(nodo1 != nodo2){
                //cout << "asociar(" << nodo1 << "," << nodo2 << ")" << endl;

                /* Agrego la arista faltante  */
                this->aristas.push_back(arista(nodo1,nodo2));

                //agrego cada nodo en la lista del otro nodo incidente

                (this->lista_global[nodo1]).push_back(nodo2);
                (this->lista_global[nodo2]).push_back(nodo1);

                (this->matriz_ady[nodo1][nodo2]) = 1;
                (this->matriz_ady[nodo2][nodo1]) = 1;
            }
        }


        int grado(int nodo) const{
            return this->lista_global[nodo].size();
        }

        bool hayArista(int nodo1, int nodo2) const{
            return (matriz_ady[nodo2][nodo1] == 1);
        }

        // Devuelve un vector de int con los vecinos del nodo 'n'
        vector<int> vecinos(int n){
            vector<int> vecinos;
            //cout << "vecino de " << n + 1 << ": ";
            for (int i = 0; i < (this->lista_global[n]).size(); ++i){
                //cout  << (*this->lista_global[n])[i] << " ";
                vecinos.push_back((this->lista_global[n])[i]);
            }
            //cout << endl;
            return vecinos;
        }

        int nodoDeGradoMaximo(){
            int nodoMax = -1;
            int gradoMax = -1;
            for (int i = 0; i < cantNodos; i++ ){
                if (grado(i) > gradoMax){
                    nodoMax = i;
                    gradoMax = grado(i);
                }
            }
            return nodoMax;
        }


        /* agarro un nodo que lo paso por parametro
           visito a todos sus vecinos
           y a los que no estan marcados, los visito recursivamente y los marco
           */
        // empiezo con contador en uno, cuando es impar encontre un ciclo impar

        void devolverCicloImpar(int nodoInicial, int nodo, int nodoFinal,const vector<int> &predecesor, vector < vector <int> > &elvec, vector<bool> &marcado){

            vector<int> vec;

            int temp =  nodo;
            //cout << endl;

            //cout << nodoFinal+1 << " " ;
            vec.push_back(nodoFinal );
            marcado[nodoFinal] = true;

            int contador =0;
            while(temp != nodoFinal ){
              //  cout << temp+1 << " ";
                vec.push_back(temp);
                marcado[temp] = true;
                temp = predecesor[temp];
            }
            
            //cout << endl;
            
            elvec.push_back(vec);
      
        }

        // probar con marcado pasado por referencia y sin. y con  esa restriccion del predecesor en el if y sin

        int dfsRecur(int nodoInicial, int nodo, vector <int> &predecesor, int contador, vector< vector <int> > & vec2vec, vector<bool> &marcado){

            //cout << "me llaman con el nodo " << nodo +1<< endl;
            for (int i = 0; i < vecinos(nodo).size(); i++)
            {
                int elvecino = vecinos(nodo)[i]; 
                if (marcado[elvecino]) continue;

                if(predecesor[elvecino] == -1){

                    //cout << elvecino +1 << endl;
                    predecesor[elvecino] = nodo;
                    //contador++;
                    dfsRecur(nodoInicial, elvecino,predecesor, contador + 1, vec2vec, marcado);
                    predecesor[elvecino] = -1;



                }else{
                    if((contador % 2) == 1 ){//&& elvecino != predecesor[nodo]){    // encontre un ciclo impar
                        //cout << "el " << elvecino + 1 << " el predecor de " << nodo + 1  << " es " << predecesor[nodo] + 1 << " " <<  contador << endl;
                        devolverCicloImpar(nodoInicial, nodo, elvecino , predecesor, vec2vec, marcado);
                        return 0;
                    } 

                }

            }

            return 0;

        }


        void dfs(int nodo, vector< vector <int> > & vec2vec, vector<bool> &marcado){

            vector<int> predecesor(cantNodos,-1);
            predecesor[nodo] = -2;
            int contador = 1;
            dfsRecur(nodo, nodo, predecesor, contador, vec2vec, marcado);

        }



        int generarCliquesMaximales(vector< vector<int> > &vec2vec){  


            vector<int> vec;
            set<int> setPuestos;

            for (int i = 0; i < cantNodos; i++)
                setPuestos.insert(i);

            while( !setPuestos.empty() ){

                std::set<int>::iterator it=setPuestos.begin();

          vec = generarCliqueMaximal(*it, setPuestos);
          vec2vec.push_back(vec);
      
            }
        }


        vector<int> generarCliqueMaximal(int nodo, set<int> &setPuestos){
      
            vector<int> clique;
            set <int> seConectanConCliqueActual;

      
            for (int j = 0; j < (this->lista_global[nodo]).size(); j++){
                //cout  << "los vecinos de " <<(this->lista_global[nodo])[j] << " ";
                seConectanConCliqueActual.insert((this->lista_global[nodo])[j] );
                //cout << "agrego al nodo " << (this->lista_global[nodo])[j] +1<< endl;
                
            }
      
            clique.push_back(nodo);
            setPuestos.erase(nodo);
            
            while( !seConectanConCliqueActual.empty() ){    //mientras no es vacio, voy agregando nodos y reduciendo el set

                int elQueMasAporta;
                // agrego al que mas aporta dentro del set que se conectan con la clique


                int maximo_aporte = 0;

                for (set<int>::iterator it2= seConectanConCliqueActual.begin(); it2!=seConectanConCliqueActual.end(); ++it2){       
                    int aporta = grado(*it2);
                    //cout << "esta el" << *it << endl;
                    if (aporta > maximo_aporte){
                        elQueMasAporta = *it2;
                        maximo_aporte = aporta;
                    }
                }

                clique.push_back(elQueMasAporta);
                //cout << "Agrego al nodo " << elQueMasAporta +1 << endl;
                setPuestos.erase(elQueMasAporta);
             
        
        set<int>::iterator temp;
                // saco a todos los que no son adyacentes al nodo que acabo de agregar
                for (set<int>::iterator it= seConectanConCliqueActual.begin(); it!=seConectanConCliqueActual.end();){       
                    //cout << "aparece el nodo " << *it +1 << endl;
                        
                    if ( !hayArista(*it, elQueMasAporta)){
                       // cout << "borro el nodo " << *it+1 << " pq no se conecta con " << elQueMasAporta +1 << endl;
                        temp = it;
                        ++it;
                        seConectanConCliqueActual.erase(temp);
                        
                    }else{
            //cout << "NO borro el nodo " << *it+1 << " pq se conecta con " << elQueMasAporta +1 << endl;
            ++it;
          }
                }

            }
        
    
            return clique;
        }


    
    void cliques( vector< vector<int> > &elvec, int cantCortes){
     
      generarCliquesMaximales(elvec);
      
      vector < vector<int> > temp;
            
            sort (elvec.begin(), elvec.end(), funcionSort); 
            
            if(cantCortes > elvec.size())
            {
                cout << "la cantidad de cortes que pasaron por parametros es mayor a la cantidad total de cliques que encontre" << endl;
 
                cantCortes = elvec.size();
            }

            for (int k = 0; k < cantCortes; k++)
            {
                temp.push_back(elvec[k]);
            }

            cout << "hago " << cantCortes << " cliques" << endl;
            
            elvec = temp;
      
    }

    vector <int> heuristica(){
    
    vector< vector<int> > elvec;
    
    //genero el grafo complemento
    grafo gComplemento = grafo(0) ;
    
    gComplemento.constructor(cantNodos);
    
    for (int i = 0; i < cantNodos; i++)
    {
      for (int j = i+1; j < cantNodos; j++)
      {
        if(!hayArista(i,j)){
          gComplemento.asociar(i+1,j+1);
        }
      }
    }
    
    gComplemento.cliques(elvec,10000);
    
    vector<int> res;
    cout << endl << "el conjunto independiente: " << endl;
    for (int i = 0; i < elvec[0].size(); i++)
    {
      res.push_back(elvec[0][i]);
      cout << elvec[0][i] << endl;
    }
    
    cout << "en total tiene " << res.size() <<" nodos." << endl;
    
    return res;
    }

    void ciclosImpares( vector< vector<int> > &elvec, int cantCortes){
     
      vector<bool> marcado(cantNodos,false);
      
      for (int k = 0; k < cantNodos; k++)
      {
        dfs(k, elvec, marcado);
      }
      
      vector < vector<int> > temp;
      
      sort (elvec.begin(), elvec.end(), funcionSort); 
      
            if(cantCortes > elvec.size())
            {
                cout << "la cantidad de cortes que pasaron por parametros es mayor a la cantidad total de ciclos que encontre" << endl;
                cantCortes = elvec.size();
            }

      for (int k = 0; k < cantCortes; k++)
      {
        temp.push_back(elvec[k]);
      }

            
      elvec = temp;
      
    }
       
    private:
        void constructor(int n) {
            this->cantNodos = n;

            vector <int> vec;

            //inicializo todas las listas vacias
            for(int i = 0; i < n; i++){
                (this->lista_global).push_back(vector <int>()) ;
            }

            for(int i = 0; i < n; i++){
                (this->matriz_ady).push_back(vector <int>(n)) ;
            }
        }

        };

\end{lstlisting}

